
\section{Theoretical results}

In this section, we study the behavior of $T_{LRT}$.

Although $T_{LRT}$ is derived under~\eqref{andersonAssumption}, we study the behavior of $T_{LRT}$ in a more general setting.

\begin{assumption}\label{newDataAssumption}
    Suppose $Z_{ki}$'s are i.i.d. random vectors with common distribution $N_p(0,I_p)$, $i=1,\ldots,n_k$, $k=1,2$.
    Let $X_{ki}=U_k\Lambda_k^{1/2}Z_{ki}$, where $\Lambda_k=\mydiag(\lambda_1,\ldots,\lambda_p)$ with $\lambda_{k1}\geq \cdots \geq \lambda_{kp}$ and $U_k$ is a $p\times p$ orthgonal matrix, $i=1,\ldots,n_k$, $k=1,2$.
    Suppose $c\leq \lambda_{kp} \leq \lambda_{k,r_k+1}\leq C$, where $c>0$ and $C>0$ are absolute constants.
\end{assumption}
    Let $V_k$ and $\tilde{V}_k$ be the first $r_k$ columns and last $p-r_k$ columns of $U_k$ respectively, $k=1,2$.
    Let $\Lambda_{k,(1)}=\mydiag(\lambda_{k1},\ldots,\lambda_{kr_k})$ and $\Lambda_{k,(2)}=\mydiag(\lambda_{k,r+1},\ldots,\lambda_{kp})$, $k=1,2$.

Write
$$
T_{LRT}=\frac{1}{\tau}\sum_{i=1}^r \hat{\lambda}_i^{-1}\big(\hat{u}_i^T(\bar{X}_1-\bar{X}_2)\big)^2
+\frac{1}{\tau\hat{\sigma}^{2}} \|\hat{\tilde{V}}^T (\bar{X}_1-\bar{X}_2)\|^2
\overset{def}{=}T_1 + \hat{\sigma}^{-2}T_2.
$$
The randomness of $T_{LRT}$ is mainly due to $T_2$.
$T_1$ can be regard as a power enhancement term, see~\cite{Fan2015Power}.
Hence we deal with $T_1$ and $T_2$ separately.
The following theorem gives the asymptotic and nonasymptotic results for $T_2$.
\begin{theorem}\label{xxxTheorem}
    Suppose the null hypothesis holds.
    Let $r$ be fixed as $n,p$ vary.
    Denote by $Q(1-\alpha)$ the $1-\alpha$ quantile of $T_2-(p-r)\hat{\sigma}^2$.
    Then
    \begin{enumerate}[(a)]
        \item
            \begin{equation*}
    Q(1-\alpha)\lesssim 
        \sqrt{p}+\frac{p}{n}+\frac{\lambda_1\max(p,n)}{\max(n\lambda_r,\max(p,n))}.
            \end{equation*}
        %\item
            %If $p/n\to \infty$, $p/(n\lambda_1)\to 0$ and ${p}/n^2\to \infty$, then $Q^*(1-\alpha)$, the $1-\alpha$ quantile of $T_2-\sum_{i=r+1}^{p}\lambda_i$, satisfies
            %\begin{equation*}
    %Q(1-\alpha)\gtrsim 
                %\frac{p}{n}.
            %\end{equation*}
        \item
        Suppose $n,p\to \infty$ and
    \begin{equation}\label{nullTheorem:eq4}
        \frac{p}{n^2}\to 0,\quad \quad
        \frac{\lambda_1\max(p,n)}{\sqrt{p}\max\big(n\lambda_r,\max(p,n)\big)}\to 0.
        %\frac{p}{n\lambda_r}=O(1),
        %\quad\quad
        %\frac{\max(p,n)\lambda_1}{n\sqrt{p}\lambda_r}\to 0,
    \end{equation}
        then
        $$\frac{T_2-(p-r)\hat{\sigma}^2}{\sqrt{2\sum_{i=r+1}^p \lambda_i^2}}\xrightarrow{\mathcal{L}}N(0,1).$$
    \end{enumerate}



\end{theorem}
\begin{remark}

    If $\lambda_1\asymp\lambda_r$, then~\eqref{nullTheorem:eq4} is equivalent to $p/n^2\to 0$.
In this case, it can be shown that $p/n^2\to 0$ is the necessary condition for asymptotic normality
\end{remark}
\begin{proof}


    Under the null hypothesis
    $$
    \hat{\tilde{V}}^T (\bar{X}_1-\bar{X}_2)=\hat{\tilde{V}}^T U \Lambda^{1/2}(\bar{Z}_1-\bar{Z}_2).
    $$
    Hence
    $$
    T_2=\frac{1}{\tau}\|\hat{\tilde{V}}^T U \Lambda^{1/2}(\bar{Z}_1-\bar{Z}_2)\|^2.
    $$

    By the property of normal random variables, $\{\bar{Z_1},\bar{Z_2}\}$ are independent of $S$.
    Since $\hat{\tilde{V}}$ only depends on $S$, it is independent of $\{\bar{Z_1},\bar{Z_2}\}$.
    It can be easily shown that
    \begin{equation}\label{nullTheorem:twoTerms}
    \begin{split}
        \frac{1}{\tau}\|\hat{\tilde{V}}^T U \Lambda^{1/2}(\bar{Z}_1-\bar{Z}_2)\|^2\sim&
    \sum_{i=1}^{p-r} \lambda_i (\hat{\tilde{V}}^T U \Lambda U^T \hat{\tilde{V}}) \xi_i^2\\
        =&
         \sum_{i=1}^{r} \lambda_i (\hat{\tilde{V}}^T U \Lambda U^T \hat{\tilde{V}}) \xi_i^2
    +
     \sum_{i=r+1}^{p-r} \lambda_i (\hat{\tilde{V}}^T U \Lambda U^T \hat{\tilde{V}}) \xi_i^2
    ,
    \end{split}
    \end{equation}
    where $\{\xi_{i}\}_{i=1}^{p-r}$ are independent standard normal random variables and are independent of $\hat{\tilde{V}}$.


    First we deal with 
        $\sum_{i=1}^{r} \lambda_i (\hat{\tilde{V}}^T U \Lambda U^T \hat{\tilde{V}}) \xi_i^2$.
    By writting out
    $$
    U\Lambda U^T
    =
    V\Lambda_{(1)}V^T+
    \tilde{V}\Lambda_{(2)}\tilde{V}^T,
    $$
    we have matrix inequality
    $$
    \lambda_r VV^T 
    \leq
    U\Lambda U^T
    \leq \lambda_1 VV^T +CI_p.
    $$
    Thus, for $i= 1,\ldots,r$, we have
    $$
    \lambda_r \lambda_i(\hat{\tilde{V}}^T U U^T \hat{\tilde{V}})
    \leq
    \lambda_i(\hat{\tilde{V}}^T U\Lambda U^T \hat{\tilde{V}})
    \leq
    \lambda_1 \lambda_i(\hat{\tilde{V}}^T V V^T \hat{\tilde{V}})+C.
    $$
Then $
    \sum_{i=1}^{r} \lambda_i (\hat{\tilde{V}}^T U \Lambda U^T \hat{\tilde{V}}) \xi_i^2
    $
    can be upper and lower bounded
$$
    \lambda_r \lambda_r(\hat{\tilde{V}}^T V V^T \hat{\tilde{V}})
    \sum_{i=1}^{r} 
    \xi_i^2
\leq
    \sum_{i=1}^{r} \lambda_i (\hat{\tilde{V}}^T U \Lambda U^T \hat{\tilde{V}}) \xi_i^2
    \leq
    (\lambda_1 \lambda_1(\hat{\tilde{V}}^T V V^T \hat{\tilde{V}})+C)
    \sum_{i=1}^{r} 
    \xi_i^2.
$$


Now by Theorem~\ref{theoremEigenvectors} and the independence of $\hat{\tilde{V}}$ and $\xi_i$'s, with probability at least $1-3\exp(-t^2/2)$, we have
    \begin{equation}\label{nullTheorem:eqLong1}
    \sum_{i=1}^{r} \lambda_i (\hat{\tilde{V}}^T U \Lambda U^T \hat{\tilde{V}}) \xi_i^2
    \leq
\big(\frac{C\lambda_1}{n\hat{\lambda}_r}(\sqrt{p-r}+\sqrt{n-2}+t)^2+C\big) q_1(r,t)
,
    \end{equation}
     where $q_1(r,t)$ is the $1-\exp(-t^2/2)$ quantile of $\chi^2_r$.
 Theorem~\ref{PCATheoremEigenvalue} can lower bound $\hat{\lambda}_r$ which appears in the right hand side of~\eqref{nullTheorem:eqLong1}.
 More precisely,
with probability at least $1-9\exp(-t^2/2)$, we have
    \begin{equation}\label{nullTheorem:eqLong}
            {\hat{\lambda}_r}\geq 
            \nu(t,\lambda_r,n,p),
    \end{equation}
 where
\begin{equation*}
        \begin{split}
            &\nu(t,\lambda_r,n,p)=
            \max \Big(\lambda_r\big(1-\frac{2}{\sqrt{n-2}}(\sqrt{r}+t)\big),\\
            &\frac{c\max(p-r,n-2)}{n-2} \big( 1-\sqrt{\frac{2}{(p-r)(n-2)}}t-
            (r-1)\big(\frac{1}{\sqrt{p-r}}+\frac{1}{\sqrt{n-2}}+\frac{t}{\sqrt{(p-r)(n-2)}}\big)^2 \big) \Big).
        \end{split}
    \end{equation*}



As a conclusion of~\eqref{nullTheorem:eqLong1} and~\eqref{nullTheorem:eqLong}, as $n,p \to \infty$, 
\begin{equation}\label{nullTheorem:eq97}
    \sum_{i=1}^{r} \lambda_i (\hat{\tilde{V}}^T U \Lambda U^T \hat{\tilde{V}}) \xi_i^2
    =
O_P\Big(\frac{\lambda_1\max(p,n)}{\max\big(n\lambda_r,\max(p,n)\big)}\Big).
\end{equation}
%Since $\frac{p}{n\lambda_r}=O(1)$, 
%\begin{equation*}
%\frac{\lambda_1\max(p,n)}{n\lambda_r\max\big(1,\frac{\max(p,n)}{n\lambda_r}\big)}=o\Big(\sqrt{2\sum_{i=r+1}^p \lambda_i^2}\Big) 
%\quad\textrm{if and only if}\quad
%\frac{ \max(p,n)\lambda_1}{
    %n\sqrt{p}\lambda_r 
%}\to 0.
%\end{equation*}

By~\eqref{nullTheorem:eq4} and~\eqref{nullTheorem:eq97}, 

\begin{equation}\label{nullTheorem:eq3}
\sum_{i=1}^{r} \lambda_i (\hat{\tilde{V}}^T U \Lambda U^T \hat{\tilde{V}}) \xi_i^2
=o_p\Big(\sqrt{2\sum_{i=r+1}^p \lambda_i^2}\Big).
\end{equation}


%If $p-r\geq n-2$, then with probability at least $1-3\exp(-t^2/2)$, we have
%\begin{equation}\label{nullTheorem:mei1}
    %\sum_{i=1}^{r} \lambda_i (\hat{\tilde{V}}^T U \Lambda U^T \hat{\tilde{V}}) \xi_i^2
%\geq
%\frac{c\lambda_r}{n\hat{\lambda}_1}(\sqrt{p-r}-\sqrt{n-2}-t)^2
%q_0(r,t)
%,
%\end{equation}
%where $q_0(r,t)$ is the $\exp(-t^2/2)$ quantile of $\chi^2_r$.
%And $\hat{\lambda}_1$ can be also upper bounded by Theorem~\ref{PCATheoremEigenvalue}.



    %For $i=r+1,\ldots,p-r$, $\lambda_i(\hat{\tilde{V}}^T U\Lambda U^T \hat{\tilde{V}})$
  Turning to the second term of~\eqref{nullTheorem:twoTerms}.
Since $c\leq \lambda_i\leq C$, $i=r+1,\ldots,p$. By Lemma~\ref{PCAlemma1}, with probability at least $1-\exp(-t^2/2)$, we have
    \begin{equation}\label{nullTheorem:eqmm1}
    \sum_{i=r+1}^{p-r} \lambda_i (\hat{\tilde{V}}^T U \Lambda U^T \hat{\tilde{V}}) \xi_i^2
    \leq \sum_{i=r+1}^{p-r} \lambda_i (\hat{\tilde{V}}^T U \Lambda U^T \hat{\tilde{V}}) +\sqrt{2p}Ct + Ct^2.
    \end{equation}
%%On the other hand, with probability at least $1-\exp(-t^2/2)$, we have
    %\begin{equation}\label{nullTheorem:eqnn}
    %\sum_{i=r+1}^{p-r} \lambda_i (\hat{\tilde{V}}^T U \Lambda U^T \hat{\tilde{V}}) \xi_i^2
    %\geq \sum_{i=r+1}^{p-r} \lambda_i (\hat{\tilde{V}}^T U \Lambda U^T \hat{\tilde{V}}) -\sqrt{2p}ct.
    %\end{equation}
  
  For $i= r+1,\ldots,p-r$, we have that
    $$
    \lambda_i(\hat{\tilde{V}}^T U\Lambda U^T \hat{\tilde{V}})
    =
    \lambda_i(\Lambda^{1/2} U^T \hat{\tilde{V}}\hat{\tilde{V}}^T U\Lambda^{1/2})
    =
    \lambda_i(\Lambda-\Lambda^{1/2} U^T\hat{V}\hat{V}^T U\Lambda^{1/2}).
    $$
    Since $\myrank(\Lambda^{1/2} U^T\hat{V}\hat{V}^T U\Lambda^{1/2})=r$, by Weyl's inequality, $\lambda_i(\hat{\tilde{V}}^T U\Lambda U^T \hat{\tilde{V}})$ can be bounded by
    \begin{equation}\label{nullTheorem:Weyl}
    \lambda_{i+r}
    \leq \lambda_i(\hat{\tilde{V}}^T U\Lambda U^T \hat{\tilde{V}})\leq \lambda_i.
    \end{equation}

By~\eqref{nullTheorem:Weyl}, $\sum_{i=r+1}^p \lambda_i -rC\leq \sum_{i=r+1}^{p-r} \lambda_i (\hat{\tilde{V}}^T U \Lambda U^T \hat{\tilde{V}})\leq \sum_{i=r+1}^p \lambda_i$, then~\eqref{nullTheorem:eqmm1} implies
    \begin{equation}\label{nullTheorem:eqmm}
    \sum_{i=r+1}^{p-r} \lambda_i (\hat{\tilde{V}}^T U \Lambda U^T \hat{\tilde{V}}) \xi_i^2
    \leq \sum_{i=r+1}^p \lambda_i+\sqrt{2p}Ct + Ct^2,
    \end{equation}
    %and~\eqref{nullTheorem:eqnn} implies
    %\begin{equation}\label{nullTheorem:eqnn1}
    %\sum_{i=r+1}^{p-r} \lambda_i (\hat{\tilde{V}}^T U \Lambda U^T \hat{\tilde{V}}) \xi_i^2
    %\geq \sum_{i=r+1}^p \lambda_i -\sqrt{2p}ct-rC.
    %\end{equation}



As another conclusion of~\eqref{nullTheorem:Weyl},
    $$
    \sum_{i=r+1}^{p-r}\lambda_{i+r} \xi_i^2   \leq
    \sum_{i=r+1}^{p-r} \lambda_i (\hat{\tilde{V}}^T U \Lambda U^T \hat{\tilde{V}}) \xi_i^2
    \leq
\sum_{i=r+1}^{p-r}\lambda_i \xi_i^2.
    $$
By Lyapunov's central limit theorem, as $p\to \infty$,
$$
    \frac{
        \sum_{i=r+1}^{p-r} \lambda_i (\hat{\tilde{V}}^T U \Lambda U^T \hat{\tilde{V}}) \xi_i^2
    -\sum_{i=r+1}^{p}\lambda_i}{
    \sqrt{2\sum_{i=r+1}^p\lambda_i^2}
    }
    \xrightarrow{\mathcal{L}}N(0,1).
    $$
    Together with~\eqref{nullTheorem:eq3}, we can deduce that under~\eqref{nullTheorem:eq3},
\begin{equation}\label{nullTheorem:eq2}
\begin{split}
    \frac{
        \tau^{-1}\|\hat{\tilde{V}}^T U \Lambda^{1/2}(\bar{Z}_1-\bar{Z}_2)\|^2
        -\sum_{i=r+1}^p \lambda_i
    }
    {
    \sqrt{2\sum_{i=r+1}^p\lambda_i^2}
    }
    \xrightarrow{\mathcal{L}}N(0,1).
\end{split}
\end{equation}
Next we deal with $\hat{\sigma}^2$.
By Theorem~\ref{PCAsigma}, with probability at least $1-4\exp(-t^2/2)$, we have
\begin{equation}\label{nullTHeorem:eq10}
\Big|
\frac{\hat{\sigma}^2}{(p-r)^{-1}\sum_{i=r+1}^p \lambda_i}-1
\Big|
\leq
\frac{1}{(n-2)\sum_{i=r+1}^p \lambda_i}
\big(
rC(\sqrt{p-r}+\sqrt{n-2}+t)^2+\sqrt{2(n-2)p}Ct+Ct^2
\big).
\end{equation}
It follows that
\begin{equation}\label{nullTheorem:eq1}
\hat{\sigma}^2
    =\frac{\sum_{i=r+1}^p \lambda_i}{p-r}+O_p(\frac{1}{\min(n,p)}).
\end{equation}
By~\eqref{nullTheorem:eq2},~\eqref{nullTheorem:eq1},~\eqref{nullTheorem:eq4} and Slutsky's theorem, we have
\begin{equation}
\begin{split}
    &\frac{
        \tau^{-1}\|\hat{\tilde{V}}^T U \Lambda^{1/2}(\bar{Z}_1-\bar{Z}_2)\|^2
        -(p-r)\hat{\sigma}^2
    }
    {
    \sqrt{2\sum_{i=r+1}^p\lambda_i^2}
    }\\
    =&
    \frac{
        \tau^{-1}\|\hat{\tilde{V}}^T U \Lambda^{1/2}(\bar{Z}_1-\bar{Z}_2)\|^2
        -
    \sum_{i=r+1}^p \lambda_i
    }
    {
    \sqrt{2\sum_{i=r+1}^p\lambda_i^2}
    }
    +O_p(\frac{\sqrt{p}}{\min(n,p)})
    \xrightarrow{\mathcal{L}}N(0,1).
\end{split}
\end{equation}
Conclusion $(b)$ is established.



    With probability at least $1-17\exp(-t^2/2)$, inequality~\eqref{nullTheorem:eqLong1},~\eqref{nullTheorem:eqLong},~\eqref{nullTheorem:eqmm} and~\eqref{nullTHeorem:eq10} are satisfied simultaneously.
    Choose $t^*$ such that $17\exp(-t^{*2}/2)=\alpha$.
    Then~\eqref{nullTheorem:eqLong1},~\eqref{nullTheorem:eqLong},~\eqref{nullTheorem:eqmm} and~\eqref{nullTHeorem:eq10} imply that
\begin{equation*}
    \begin{split}
        Q(1-\alpha)&\leq
\sum_{i=r+1}^p \lambda_i+\sqrt{2p}Ct^* +C t^{*2}
    +
\big(\frac{C\lambda_1}{n\nu(t^*,\lambda_r,n,p)}(\sqrt{p-r}+\sqrt{n-2}+t^*)^2+C\big) q_1(r,t^*)\\
        &-\Big(\sum_{i=r+1}^p \lambda_i-
        \frac{1}{n-2}
\big(
rC(\sqrt{p-r}+\sqrt{n-2}+t)^2+\sqrt{2(n-2)p}Ct+Ct^2
        \big)\Big)\\
        %&\lesssim
        &\asymp
        \sqrt{p}+\frac{p}{n}+\frac{\lambda_1\max(p,n)}{n \nu(t^*,\lambda_r,n,p)}
        \asymp
        \sqrt{p}+\frac{p}{n}+\frac{\lambda_1\max(p,n)}{\max(n\lambda_r,\max(p,n))}.
    \end{split}
\end{equation*}
Conclusion $(a)$ is established.

%Conclusion $(b)$ can be proved similarly.
    %Choose $t^*$ such that $17\exp(-t^{*2}/2)=1-\alpha$.
 %By~\eqref{nullTheorem:mei1}, Theorem~\ref{PCATheoremEigenvalue} and~\eqref{nullTheorem:eqnn1}, we have
%$$
%\begin{aligned}
    %Q(1-\alpha)\geq&
%\sum_{i=r+1}^p \lambda_i-\sqrt{2p}Ct^* -rC
    %+
%\frac{C\lambda_r(\sqrt{p-r}-\sqrt{n-2}-t^*)^2}{n\nu'(t^*,\lambda_1,n,p)} q_1(r,t^*)-\sum_{i=r+1}^p \lambda_i\\
%\end{aligned}
%$$
%where 
%$$
%\begin{aligned}
    %\nu'(t^*,\lambda_1,n,p)&=\lambda_1(1+\frac{2}{\sqrt{n-2}}(\sqrt{r}+t^*)+\frac{1}{n-2}(\sqrt{r}+t^*)^2+\frac{C}{(n-2)\lambda_1}(\sqrt{p-r}+\sqrt{n-2}+t^*)^2)
    %\\
    %&\asymp
    %\lambda_1(1+\frac{p}{n \lambda_1}).
%\end{aligned}
%$$
%Hence
%$$
%\begin{aligned}
    %Q(1-\alpha)\gtrsim& -\sqrt{p}+\frac{\lambda_r p}{n\lambda_1(1+p/(n\lambda_1))}\asymp \frac{p}{n}.
%\end{aligned}
%$$



\end{proof}

\begin{theorem}
    Let $r$ be fixed as $n,p\to\infty$.
    Suppose $np^{-1/2}\|\mu_1-\mu_2\|^2=O(1)$, ${\lambda_1}/{\sqrt{p}}\to \infty$ and~\eqref{nullTheorem:eq4} holds.
    Then
        $$\frac{T_2-(p-r)\hat{\sigma}^2-\tau^{-1}\|\tilde{V}^T(\mu_1-\mu_2)\|^2}{\sqrt{2\sum_{i=r+1}^p \lambda_i^2}}\xrightarrow{\mathcal{L}}N(0,1).$$
\end{theorem}
\begin{proof}

    Since
    $$
    \hat{\tilde{V}}^T (\bar{X}_1-\bar{X}_2)=\hat{\tilde{V}}^T U \Lambda^{1/2}(\bar{Z}_1-\bar{Z}_2)+\hat{\tilde{V}}^T (\mu_1-\mu_2),
    $$
    $T_2$ can be written as the sum of three terms
    $$
    T_2=\tau^{-1}\|\hat{\tilde{V}}^T U \Lambda^{1/2}(\bar{Z}_1-\bar{Z}_2)\|^2+
 2\tau^{-1}(\mu_1-\mu_2)^T\hat{\tilde{V}}\hat{\tilde{V}}^T U \Lambda^{1/2}(\bar{Z}_1-\bar{Z}_2)
+
    \tau^{-1}\|\hat{\tilde{V}}^T (\mu_1-\mu_2)\|^2.
    $$
The first term has  normal distribution asymptotically.
    We now deal with the second term.
    Note that
    $$\tau^{-1}(\mu_1-\mu_2)^T\hat{\tilde{V}}\hat{\tilde{V}}^T U \Lambda^{1/2}(\bar{Z}_1-\bar{Z}_2)\sim
    \sqrt{\tau^{-1} (\mu_1-\mu_2)^T\hat{\tilde{V}}\hat{\tilde{V}}^T U\Lambda U^T\hat{\tilde{V}}\hat{\tilde{V}}^T(\mu_1-\mu_2)}
    \xi,
    $$
    where $\xi$ is a standard normal variable and is independent of $\hat{\tilde{V}}$.
    Similar to the prove of~\eqref{nullTheorem:eq3},
    \begin{equation}\label{buxiangxie2}
        \begin{split}
            &\sqrt{\tau^{-1} (\mu_1-\mu_2)^T\hat{\tilde{V}}\hat{\tilde{V}}^T U\Lambda U^T\hat{\tilde{V}}\hat{\tilde{V}}^T(\mu_1-\mu_2)}
    \leq
            \sqrt{\tau^{-1}} \|\mu_1-\mu_2\|    
    \sqrt{\lambda_{\max}(\hat{\tilde{V}}^T U\Lambda U^T\hat{\tilde{V}})}\\
            \leq&
            \sqrt{\tau^{-1}} \|\mu_1-\mu_2\|    
    \sqrt{\lambda_1\lambda_{\max}(\hat{\tilde{V}}^T VV^T \hat{\tilde{V}})+C}
            =o_P(n^{1/2}p^{1/4}\|\mu_1-\mu_2\|)
            = o_P(\sqrt{p}).
        \end{split}
    \end{equation}

Next we deal with $    \tau^{-1}\|\hat{\tilde{V}}^T (\mu_1-\mu_2)\|^2$. Note that
    \begin{equation}\label{buxiangxie}
        \begin{split}
            &\Big|\tau^{-1}\|\hat{\tilde{V}}^T (\mu_1-\mu_2)\|^2-\tau^{-1}\|\tilde{V}^T (\mu_1-\mu_2)\|^2\Big|\\
            =&\tau^{-1}\Big|(\mu_1-\mu_2)^T(\hat{\tilde{V}}\hat{\tilde{V}}^T -\tilde{V}\tilde{V}^T) (\mu_1-\mu_2)\Big|\\
            \leq&
            \tau^{-1}\|\hat{\tilde{V}}\hat{\tilde{V}}^T -\tilde{V}\tilde{V}^T\| \|\mu_1-\mu_2\|^2
            =\tau^{-1}\|\hat{V}\hat{V}^T -VV^T\| \|\mu_1-\mu_2\|^2\\
            \leq&\tau^{-1}\|\hat{V}\hat{V}^T -VV^T\|_F \|\mu_1-\mu_2\|^2
            =\tau^{-1}\sqrt{2\mytr(\tilde{V}^T\hat{V}\hat{V}^T \tilde{V})} \|\mu_1-\mu_2\|^2\\
            \leq& \tau^{-1}\sqrt{2r\lambda_{\max}(\tilde{V}^T\hat{V}\hat{V}^T \tilde{V})} \|\mu_1-\mu_2\|^2.
        \end{split}
    \end{equation}
Condition ${\lambda_1}/{\sqrt{p}}\to \infty$ and~\eqref{nullTheorem:eq4} require
$$
        \frac{\max(p,n)}{\max\big(n\lambda_r,\max(p,n)\big)}\to 0,
    $$
    which in turn implies $\lambda_{\max}(\tilde{V}^T\hat{V}\hat{V}^T \tilde{V})\to 0$ in probability by Theorem~\ref{theoremEigenvectors}.
    Then~\eqref{buxiangxie} implies
    \begin{equation}\label{buxiangxie1}
        \begin{split}
            \tau^{-1}\|\hat{\tilde{V}}^T (\mu_1-\mu_2)\|^2=\tau^{-1}\|\tilde{V}^T (\mu_1-\mu_2)\|^2+o_P(\sqrt{p}).
        \end{split}
    \end{equation}
    The theorem follows by Theorem~\ref{xxxTheorem},~\eqref{buxiangxie2},~\eqref{buxiangxie1} and Slutsky's theorem.

\end{proof}


 



The asymptotic normality of $T_2$ requires condition~\eqref{nullTheorem:eq4}, which is strong.
However, it may not be able to be relaxed.
For simplicity, suppose $\lambda_1\asymp \lambda_r$. Then~\eqref{nullTheorem:eq4} is equivalent to $p/n^2\to 0$.
The asymptotic normality of $\|\hat{\tilde{V}}^T(\bar{X}_1-\bar{X}_2)\|^2$, the main major part of $T_2$, requires 
\begin{equation}\label{eqREML}
    \frac{\lambda_1\big(\big(\hat{\tilde{V}}^T \Sigma \hat{\tilde{V}}\big)^2\big)}{\mathrm{tr}\big(\big(\hat{\tilde{V}}^T \Sigma \hat{\tilde{V}}\big)^2\big)
}\xrightarrow{P} 0.
\end{equation}
See Lemma~\ref{quadraticFormCLT} in appendix. And~\eqref{eqREML} is equivalent to $p/n^2\to 0$ by Lemma~\ref{conRateLemma} in appendix.




%\begin{remark}  Compared with~\cite{2016arXiv160202491A}'s assumption (ix) which is equivalent to assuming $\frac{p^{2\beta-1}}{n_1+n_2}\to 0$ in model~\eqref{theModel}, our assumption $\frac{\sqrt{p}}{n_1+n_2}\to 0$ doesn't involved $\beta$.
%And when $\beta\geq \frac{3}{4}$, our assumption is weaker than~\cite{2016arXiv160202491A}'s. Note that when $\beta=\frac{1}{2}$, $\frac{\sqrt{p}}{n_1+n_2}$ is a necessary condition to make $\hat{V}\hat{V}^T$ a consistent
%estimator of $VV^T$ (see lemma 2 in appendix). So condition $\frac{\sqrt{p}}{n_1+n_2}$ is roughly the best we can do if the relationship between $p$ and $n$ doesn't involve $\beta$.
%\end{remark}

By Proposition~\ref{varianceEstimation}  and Theorem~\ref{myPanpan}, the power function of the new test can be obtained immediately.


\begin{corollary}\label{testPowerh}
    Under Assumptions~\ref{balance}-\ref{pAndN},
    if we reject the null hypothesis when $Q$ is larger than $1-\alpha$ quantile of $N(0,1)$, then the asymptotic power function of the new test is
    \begin{equation*}
        \Phi\Big(-\Phi^{-1}(1-\alpha)+\frac{\|\tilde{V}(\mu_1-\mu_2)\|^2}{\sigma^2\sqrt{2\tau^2p}}\Big).
    \end{equation*}
\end{corollary}


 Note that the power of $T_{CQ}$ is of the form
\begin{equation*}
    \Phi\Big(-\Phi^{-1}(1-\alpha)+\frac{\|\mu_1-\mu_2\|^2}{\sqrt{2\tau^2\mathrm{tr}\Sigma^2}}\Big).
\end{equation*}
 The relative efficiency of our test with respect to Chen's test is
\begin{equation*}
    \sqrt{\frac{\mathrm{tr}\Sigma^2}{(p-r)\sigma^4}}\frac{\|\tilde{V}(\mu_1-\mu_2)\|^2}{\|\mu_1-\mu_2\|^2}\sim p^{\beta-1/2}\frac{\|\tilde{V}(\mu_1-\mu_2)\|^2}{\|\mu_1-\mu_2\|^2},
\end{equation*}
which is large when $\beta>1/2$ and $\|\tilde{V}(\mu_1-\mu_2)\|/\|\mu_1-\mu_2\|$ is close to $1$.


However, this does not mean the new statistic can not be used.
In fact, since the samples are exchangeable under null hypothesis, we can always use permutation method to determine the critical value.
We will see from simulation results that the new test has good power behavior even in large $p$ small $n$ case.



In practice, it may not be an easy task to check if the covariance matrices are spiked, especially in high dimension setting.
When the spiked covariance model is not valid,
some estimators in our test procedure make no sense.
In particular, if $\hat{r}$ is estimated by~\eqref{estimateR}.
the $\hat{r}$ is nothing but a random integer not greater than $R$ and $\hat{V}\hat{V}^T$ is just a random projection.
Hence it is a natural question how the new test procedure behaves when the spiked covariance model breaks down.
We study the asymptotic behavior of the new test procedure in two non-spiked setting.

First we consider the case  when the eigenvalues of $\Sigma$ is bounded.
%In many practical problems, the alternative is `dense', i.e., under $H_1$ the signals in $\mu_1-\mu_2$ spread out over a large number of co-ordinates. See~\cite{Tony2013}.
Similar to bayesian models, we assume a normal prior distribution for $\mu_k$ to characterize `dense' alternative.
%When the eigenvalues of $\Sigma$ is bounded, spike variance model is not valid.
%In this case, the difference of our test statistic and~\cite{Chen2010A}'s is small.
The next theorem shows that  the power of our new test is asymptotically the same as~\cite{Chen2010A}'s test in this case.


\begin{theorem}\label{sameTheorem}
   Assume $X_{ki}\sim N(\mu_k,\Sigma)$,  $i=1,\ldots,n_k$, $k=1,2$.
    Suppose that Assumptions~\ref{balance} and~\ref{pAndN} holds, $0<c\leq\lambda_p(\Sigma)\leq\lambda_1(\Sigma)\leq C<\infty$ where $c$ and $C$ are constants, each element of $\mu_k$ is independently generated by $N(0,{(n_k\sqrt{p})}^{-1}\psi)$ for $k=1,2$, where $\psi$ is a constant and  $\hat{r}\leq R$ for a positive constant $R$.
    Then we have
    
\begin{equation*}
    \frac{T_2-\|\mu_1-\mu_2\|^2}{\sqrt{2\tau^2 \mathrm{tr}\Sigma^2}} \xrightarrow{\mathcal{L}} N(0,1).
\end{equation*}
\end{theorem}

The second setting we consider is the model in Assumption~\ref{theModel} with $r=0$.
In this case, the Assumption~\ref{pAndN} can be dropped and we don't need to assume a random $\mu_k$.

\begin{theorem}\label{sameTheorem2}
    Under Assumptions~\ref{balance}-\ref{theModel2} with factor number $r=0$, if
    $$
    \frac{n}{\sqrt{p}}\|\mu_1-\mu_2\|^2=O(1),
    $$
    and $\hat{r}\leq R$ for a positive constant $R$,
    then
    $$
    \frac{T_2-\|\mu_1-\mu_2\|^2}{\sigma^2\sqrt{2\tau^2 p}}\xrightarrow{\mathcal{L}} N(0,1).
    $$
\end{theorem}
These results show that the new test procedure is robust against the invalidity of spiked covariance model.


