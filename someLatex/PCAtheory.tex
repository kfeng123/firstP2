\subsection{PCA Theory}

In this section, we give some non-asymptotic theory of PCA.
Since our main task is to obtain the properties of principal space,
we impose less assumptions compared with existing results.


%The following perturbation bound for symmetric matrix, established by~\cite{MR0264450}, is known as $\sin \theta$ theorem. The current form is from~\cite{Cai2012Sparse}.
%\begin{lemma}[$\sin \theta$ theorem for symmetric matrices]
%Let $A$ and $A+E$ be symmetric matrices satisfying
%$$
    %A=(F_0, F_1)
    %\begin{pmatrix}A_0&0\\0&A_1\end{pmatrix}
    %\begin{pmatrix}
    %F_0^T
    %\\
    %F_1^T
    %\end{pmatrix},\quad
    %A+E=(G_0, G_1)
    %\begin{pmatrix}\Lambda_0&0\\0&\Lambda_1\end{pmatrix}
    %\begin{pmatrix}
    %G_0^T
    %\\
    %G_1^T
    %\end{pmatrix},
%$$
    %where $(F_1,F_1)$ and $(G_0,G_1)$ are orthogonal matrices. If the eigenvalues of $A_0$ are contained in an interval $(a,b)$, and the eigenvalues of $\Lambda_1$ are excluded from the interval $(a-\delta,b+\delta)$ for some $\delta>0$, then
%$$
    %\frac{1}{\sqrt{2}}\|F_0 F_0^T - G_0 G_0^T\|_F \leq \delta^{-1}\min(\|F_1^T E G_0\|_F,\|F_0^T E G_1\|_F),
%$$
%and
%$$
    %\|F_0 F_0^T - G_0 G_0^T\|\leq \delta^{-1}\min(\|F_1^T E G_0\|,\|F_0^T E G_1\|).
%$$
%\end{lemma}



\begin{lemma}\label{PCAlemma2}
    Under the assumption of Lemma~\ref{matrixDeviation}, for every $t> 0$, with probability at least $1-3\exp(-t^2/2)$, for every $i$ such that $1\leq i \leq n$, we have
\begin{equation*}
    \begin{split}
        \lambda_{i}(ZZ^T)
        &\geq
        N \Big( 1-\sqrt{\frac{2}{Nn}}t-(i-1)(\frac{1}{\sqrt{N}}+\frac{1}{\sqrt{n}}+\frac{t}{\sqrt{Nn}})^2 \Big).
    \end{split}
\end{equation*}

\end{lemma}
\begin{proof}
    $\mytr (ZZ^T)$ is distributed as $\chi^2_{Nn}$. By Lemma~\ref{PCAlemma1}, for every $t>0$, with probability at least $1-\exp(-t^2/2)$, we have that
    $$
    \sum_{j=1}^{i-1}\lambda_j (ZZ^T) + \sum_{j=i}^{n}\lambda_j (ZZ^T)=\mytr (ZZ^T)\geq Nn(1-\sqrt{\frac{2}{Nn}}t),
    $$
    where $1\leq i \leq n$.
    Thus, with probability at least $1-\exp(-t^2/2)$, for every $i$ such that $1\leq i \leq n$, we have
\begin{equation*}
    \begin{split}
        &\lambda_{i}(ZZ^T)\geq \frac{1}{n}\sum_{j=i}^{n}\lambda_j (ZZ^T)\\
        \geq &
        \frac{1}{n} \Big( Nn(1-\sqrt{\frac{2}{Nn}}t)-\sum_{j=1}^{i-1}\lambda_j (ZZ^T)  \Big)\\
        \geq &
        \frac{1}{n} \Big( Nn(1-\sqrt{\frac{2}{Nn}}t)-(i-1)\lambda_1 (ZZ^T)  \Big).
    \end{split}
\end{equation*}
    By the above inequality and Lemma~\ref{matrixDeviation}, with probability at least $1-3\exp(-t^2/2)$, for every $i$ such that $1\leq i \leq n$, we have
\begin{equation*}
    \begin{split}
        \lambda_{i}(ZZ^T)
        &\geq
        \frac{1}{n} \Big( Nn(1-\sqrt{\frac{2}{Nn}}t)-(i-1)(\sqrt{N}+\sqrt{n}+t)^2 \Big)\\
        &=
        N \Big( 1-\sqrt{\frac{2}{Nn}}t-(i-1)(\frac{1}{\sqrt{N}}+\frac{1}{\sqrt{n}}+\frac{t}{\sqrt{Nn}})^2 \Big).
    \end{split}
\end{equation*}
\end{proof}

\begin{assumption}\label{PCAassump}
    Suppose that $Z=(Z_1,\ldots,Z_n)$ is an $p\times n$ random matrix whose entries $Z_{ij}$'s are i.i.d. standard normal random variables, $ i=1,\ldots, p$, $j=1,\ldots, n$.
    Let the sample matrix be $X=(X_1,\ldots,X_n)=U\Lambda^{1/2}Z$, where $\Lambda=\mydiag(\lambda_1,\ldots,\lambda_p)$ with $\lambda_1\geq\cdots\geq \lambda_p$ and $U$ is a $p\times p$ orthogonal matrix.
    Suppose  $c\leq \lambda_{p} \leq \lambda_{r+1}\leq C$, where $c>0$ and $C>0$ are absolute constants. 
\end{assumption}
    The sample covariance matrix is $\frac{1}{n}X X^T$.
    We denote by  $\frac{1}{n}X X^T=\hat{U}\hat{\Lambda}\hat{U}^T$ the spectral decomposition of $S$ where $\hat{\Lambda}=\mathrm{diag}(\hat{\lambda}_1,\ldots,\hat{\lambda}_p)$ and $\hat{U}$ is a orthogonal matrix.
    
    Let $u_i$ be the $i$th column of $U$, $i=1,\ldots, p$. Denote $U=(V,\tilde{V})$, where $V$ and $\tilde{V}$ are the first $r$ and last $p-r$ columns of $U$ respectively. Similarly, we define the corresponding part of $\hat{U}$ by $\hat{u}_i$, $\hat{V}$ and $\hat{\tilde{V}}$.

    The PCA theory is mainly focus on the convergence properties of $\hat{u}_i$ to it's population counterpart $u_i$.
    See~\cite{Jung2009PCA},~\cite{2012arXiv1211.2671S},~\cite{Shen2013Consistency} and~\cite{Fan2015Asymptotics} for some recent developements for PCA theory.
    Here we are mainly interested in the asymptotic properties of $\hat{V}$.
    Compared to existing results, the consistency results of $\hat{V}$ require less assumptions on the order of $\lambda_1,\ldots,\lambda_r$.

    Let $Z_{(1)}$ and $Z_{(2)}$ be the first $r$ rows and the last $p-r$ rows of $Z$ respectively. Then $Z_{(1)}$ is an $r\times n$ matrix and $Z_{(2)}$ is an $(p-r)\times n$ matrix.
    Let $\Lambda_{(1)}=\mathrm{diag}(\lambda_1,\ldots,\lambda_r)$ and $\Lambda_{(2)}=\mathrm{diag}(\lambda_{r+1},\ldots,\lambda_p)$.
    Define $\hat{\Lambda}_{(1)}$ and $\hat{\Lambda}_{(2)}$ in a similar way.


\begin{theorem}\label{PCATheoremEigenvalue}
    Suppose Assumption~\ref{PCAassump} holds.
    Let $i$ be a fixed number such that $1 \leq i\leq r$. Then for every $t>0$, with probability at least $1-9\exp(-t^2/2)$, we have
    \begin{equation*}
        \begin{split}
            &\frac{\hat{\lambda}_i}{\lambda_i}\geq 
            \max \Big(1-\frac{2}{\sqrt{n}}(\sqrt{r}+t),\\
            \frac{c\max(p-r,n)}{n\lambda_i} &\big( 1-\sqrt{\frac{2}{(p-r)n}}t-(i-1)\big(\frac{1}{\sqrt{p-r}}+\frac{1}{\sqrt{n}}+\frac{t}{\sqrt{(p-r)n}}\big)^2 \big) \Big),\\
        \end{split}
    \end{equation*}
    and
    \begin{equation*}
        \begin{split}
            \frac{\hat{\lambda}_i}{\lambda_i}\leq 
        1+\frac{2}{\sqrt{n}}(\sqrt{r}+t)+\frac{1}{n}(\sqrt{r}+t)^2+
        \frac{C}{n\lambda_i} (\sqrt{p-r}+\sqrt{n}+t)^2.
        \end{split}
    \end{equation*}

    %\begin{equation*}
       %L_1+L_2 \leq\frac{\hat{\lambda}_i}{\lambda_i}\leq U_1+U_2
    %\end{equation*}
    %where 
    %\begin{align*}
        %L_1&=1-\frac{2}{\sqrt{n}}(\sqrt{r}+t),\\
        %L_2&=\frac{c\max(p-r,n)}{n\lambda_i} \Big( 1-\sqrt{\frac{2}{(p-r)n}}t-(i-1)\big(\frac{1}{\sqrt{p-r}}+\frac{1}{\sqrt{n}}+\frac{t}{\sqrt{(p-r)n}}\big)^2 \Big),\\
        %U_1&=1+\frac{2}{\sqrt{n}}(\sqrt{r}+t)+\frac{1}{n}(\sqrt{r}+t)^2,\\
        %U_2&=\frac{C}{n\lambda_i} (\sqrt{p-r}+\sqrt{n}+t)^2.
    %\end{align*}

\end{theorem}
\begin{proof}
    The non-zero eigenvalues of $\frac{1}{n}XX^T$ are equal to that of $\frac{1}{n}X^T X$.
    And $\frac{1}{n}X^T X$ can be further written as the sum of two quantities
    \begin{equation*}
        \frac{1}{n}X^T X=\frac{1}{n}Z^T \Lambda Z=\frac{1}{n}Z_{(1)}^T \Lambda_{(1)} Z_{(1)} +\frac{1}{n}Z_{(2)}^T \Lambda_{(2)} Z_{(2)} \overset{def}{=}A+B.
    \end{equation*}
    By Weyl's inequality,
    \begin{equation}\label{PCA19}
        \frac{\max(\lambda_i(A),\lambda_i(B))}{\lambda_i}\leq\frac{\hat{\lambda}_i}{\lambda_i}\leq \frac{\lambda_i(A)}{\lambda_i} +\frac{\lambda_{\max}(B)}{\lambda_i},
    \end{equation}
    where $i=1,\ldots, r$. We deal with $\lambda_i(A)$ and $\lambda_i(B)$ separately.

    First we deal with ${\lambda_i(A)}$, $ i=1,\ldots, r$.
    By Corollary~\ref{WeylCor}, we have that
    \begin{equation*}
        \begin{aligned}
            \frac{\lambda_i(A)}{\lambda_i}&\leq
            \frac{1}{n\lambda_i}\lambda_{\max}\big(Z_{(1)}^T \mydiag(\underbrace{0,\ldots,0}_{i-1},\underbrace{\lambda_i,\ldots,\lambda_i}_{r-i+1}) Z_{(1)} \big)
            =\frac{1}{n}\lambda_{\max}(Z_{[i:r,:]}^T Z_{[i:r,:]}).
        \end{aligned}
    \end{equation*}
    Then by Lemma~\ref{matrixDeviation}, with probability at least $1-2\exp(-t^2/2)$ we have 
    \begin{equation}\label{eigenvalueTheorem:2}
    \frac{\lambda_i(A)}{\lambda_i}\leq
    \frac{1}{n}{(\sqrt{n}+\sqrt{r-i+1}+t)}^2
    \leq
    1+\frac{2}{\sqrt{n}}(\sqrt{r}+t)+\frac{1}{n}(\sqrt{r}+t)^2.
\end{equation}
    On the other hand, by Weyl's inequility, we have that
\begin{equation*}
    \begin{aligned}
        \frac{\lambda_i(A)}{\lambda_i}&\geq
        \frac{1}{n\lambda_i}\lambda_{i}\big(Z_{(1)}^T \mathrm{diag}(\underbrace{\lambda_i,\ldots,\lambda_i}_{i},\underbrace{0,\ldots,0}_{r-i}) Z_{(1)} \big)
            =\frac{1}{n}\lambda_{\max}(Z_{[1:i,:]}^T Z_{[1:i,:]}).
    \end{aligned}
\end{equation*}
    Again by Lemma~\ref{matrixDeviation}, with probability at least $1-2\exp(-t^2/2)$ we have 
\begin{equation}\label{eigenvalueTheorem:3}
        \frac{\lambda_i(A)}{\lambda_i}\geq
        \frac{1}{n}{\big(\max(\sqrt{n}-\sqrt{i}-t,0)\big)}^2
        \geq 1-\frac{2}{\sqrt{n}}(\sqrt{r}+t).
\end{equation}

    Now we deal with $\lambda_{i}(B)$. Since $\lambda_i(B)\geq \frac{c}{n} \lambda_i(Z_{(2)}^T Z_{(2)})$, by Lemma~\ref{PCAlemma2}, with probability at least $1-3\exp(-t^2/2)$ we have
\begin{equation}\label{eigenvalueTheorem:4}
    \begin{split}
        \lambda_{i}(B)
        &\geq
        \frac{c\max(p-r,n)}{n} \Big( 1-\sqrt{\frac{2}{(p-r)n}}t-(i-1)(\frac{1}{\sqrt{p-r}}+\frac{1}{\sqrt{n}}+\frac{t}{\sqrt{(p-r)n}})^2 \Big).
    \end{split}
\end{equation}
    Since $\lambda_1(B)\leq \frac{C}{n}\lambda_1(Z_{(2)}^T Z_{(2)})$, by Lemma~\ref{matrixDeviation}, with probability at least $1-2\exp(-t^2/2)$ we have
\begin{equation}\label{eigenvalueTheorem:5}
    \begin{split}
        \lambda_{\max}(B)
        &\leq
        \frac{C}{n} (\sqrt{p-r}+\sqrt{n}+t)^2.
    \end{split}
\end{equation}
    The theorem follows by~\eqref{PCA19},~\eqref{eigenvalueTheorem:2},~\eqref{eigenvalueTheorem:3},~\eqref{eigenvalueTheorem:4}, and \eqref{eigenvalueTheorem:5}.
\end{proof}



\begin{theorem}\label{theoremEigenvectors}
    Suppose Assumption~\ref{PCAassump} holds.
    For every $t>0$, with probability at least $1-2\exp(-t^2/2)$, we have
    \begin{equation}\label{PCAINeedThis}
        \lambda_1 (\tilde{V}^T \hat{V}\hat{V}^T \tilde{V})
        \leq \frac{C}{n\hat{\lambda}_r}(\sqrt{p-r}+\sqrt{n}+t)^2.
    \end{equation}
    With probability at least $1-\exp(-t^2/2)$, we have
    \begin{equation}\label{PCAtheorem102}
        r-\frac{1}{2}\|\hat{V}\hat{V}^T-VV^T\|^2_F\leq 
        \frac{\lambda_1}{\hat{\lambda}_r}
        (r+\sqrt{\frac{2r}{n}}t+\frac{1}{n}t^2).
    \end{equation}
If we further assume $p-r\geq n$, then with probability at least $1-2\exp(-t^2/2)$, we have
    \begin{equation}\label{PCAtheorem101}
    \lambda_r (\tilde{V}^T\hat{V}\hat{V}^T \tilde{V})\geq 
        \frac{c}{n\hat{\lambda}_1}(\max(\sqrt{p-r}-\sqrt{n}-t,0))^2.
    \end{equation}

\end{theorem}


\begin{proof}

    Since
    \begin{equation*}
        \frac{1}{n}XX^T=\hat{U}\hat{\Lambda}\hat{U}^T=
        \frac{1}{n}U\Lambda^{1/2}ZZ^T \Lambda^{1/2} U^T,
    \end{equation*}
    we have
    \begin{equation}\label{crucialEqInPCA1}
        \Lambda^{-1/2}U^T \hat{U}\hat{\Lambda}\hat{U}^T U\Lambda^{-1/2}=
        \frac{1}{n}ZZ^T. 
    \end{equation}
    We first prove~\eqref{PCAINeedThis}.
    It follows from~\eqref{crucialEqInPCA1} that
    \begin{equation}\label{crucialEqInPCA3}
        \Lambda^{-1/2}_{(2)}\tilde{V}^T \hat{U}\hat{\Lambda}\hat{U}^T \tilde{V}\Lambda^{-1/2}_{(2)}=
        \frac{1}{n}Z_{(2)} Z_{(2)}^T.
    \end{equation}

    The left hand side of~\eqref{crucialEqInPCA3} equals to $A+B$, where 
    $A= \Lambda^{-1/2}_{(2)}\tilde{V}^T \hat{V}\hat{\Lambda}_{(1)}\hat{V}^T \tilde{V}\Lambda^{-1/2}_{(2)}$ 
    and 
    $B= \Lambda^{-1/2}_{(2)}\tilde{V}^T \hat{\tilde{V}}\hat{\Lambda}_{(2)}\hat{\tilde{V}}^T \tilde{V}\Lambda^{-1/2}_{(2)}$.

    Obviously, we have $\lambda_1(A)\leq {n^{-1}} \lambda_1({Z}_{(2)} {Z}_{(2)}^T)$ and
    %$$
    %\frac{\hat{\lambda}_r}{C} \lambda_1 (\tilde{V}^T \hat{V}\hat{V}^T \tilde{V}) \leq \lambda_1(C)\leq \frac{\hat{\lambda}_1}{c} \lambda_1 (\tilde{V}^T \hat{V}\hat{V}^T \tilde{V}).
    %$$
    $
    \lambda_1(A)   \geq {C^{-1}}{\hat{\lambda}_r} \lambda_1 (\tilde{V}^T \hat{V}\hat{V}^T \tilde{V})
    $.
    It follows that 
    $$
    \lambda_1 (\tilde{V}^T \hat{V}\hat{V}^T \tilde{V})\leq 
    \frac{C}{n\hat{\lambda}_r} \lambda_1({Z}_{(2)}{Z}_{(2)}^T).
    $$
    Then~\eqref{PCAINeedThis} holds by Lemma~\ref{matrixDeviation}.
    

    If $p-r\geq n$, then $\mathrm{Rank}(A)=r$, $\mathrm{Rank}(B)=n-r$ and $\mathrm{Rank}(A+B)=n$.
    By Weyl's inequality,
    $$\lambda_n(A+B)\leq \lambda_r(A)+ \lambda_{n-r+1}(B)=\lambda_r(A).$$
    Thus, 
    $$
 \frac{1}{n}\lambda_n({Z}_{(2)}{Z}_{(2)}^T)   \leq \lambda_r(A)
    \leq  \frac{\hat{\lambda}_1}{c} \lambda_r(\tilde{V}^T \hat{V}\hat{V}^T \tilde{V}),
    $$
    or
    $$
      \lambda_r(\tilde{V}^T \hat{V}\hat{V}^T \tilde{V})
    \geq
    \frac{c}{n\hat{\lambda}_1}\lambda_n({Z}_{(2)}{Z}_{(2)}^T).
    $$
    Then~\eqref{PCAtheorem101} holds by Lemma~\ref{matrixDeviation}.


    Now we prove~\eqref{PCAtheorem102}.
    Equation~\eqref{crucialEqInPCA1} implies that
    \begin{equation}\label{crucialEqInPCA2}
        \Lambda^{-1/2}_{(1)}V^T \hat{U}\hat{\Lambda}\hat{U}^T V\Lambda^{-1/2}_{(1)}=
        \frac{1}{n}{Z}_{(1)}{Z}_{(1)}^T.
    \end{equation}
    Note that
    \begin{equation}
        \begin{aligned}
        \mathrm{tr}\big(\Lambda^{-1/2}_{(1)}V^T \hat{U}\hat{\Lambda}\hat{U}^T V\Lambda^{-1/2}_{(1)}\big)
            &\geq
        \mathrm{tr}\big(\Lambda^{-1/2}_{(1)}V^T \hat{V}\hat{\Lambda}_{(1)}\hat{V}^T V\Lambda^{-1/2}_{(1)}\big)
            \\
            &\geq
            \frac{\hat{\lambda}_r}{\lambda_1}\Big(r-\frac{1}{2}\|\hat{V}\hat{V}^T-VV^T\|^2_F\Big).
        \end{aligned}
    \end{equation}
Then
$$
            \Big(r-\frac{1}{2}\|\hat{V}\hat{V}^T-VV^T\|^2_F\Big)\leq
\frac{\lambda_1}{n\hat{\lambda}_r}\mytr Z_{(1)}Z_{(1)}^T.
$$
Hence~\eqref{PCAtheorem102} holds by Lemma~\ref{PCAlemma1}.

\end{proof}

\begin{theorem}\label{PCAsigma}
    Suppose Assumption~\ref{PCAassump} holds. Then, with probability at least $1-4\exp(-t^2/2)$, we have
    $$\Big|\sum_{i=r+1}^{\min(n,p)}\lambda_i(XX^T)-n\sum_{i=r+1}^p \lambda_i\Big|\leq rC (\sqrt{p-r}+\sqrt{n}+t)^2+\sqrt{2np}Ct +C t^2.$$
\end{theorem}
\begin{proof}
     We deal with $Z^T \Lambda Z$ instead of $XX^T$ since $\lambda_i(X X^T)=\lambda_i (Z^T \Lambda Z)$, $i=1\ldots,\min(n,p)$.
    And $Z^T \Lambda Z$ can be written as the sum of two quantities
    $$
 Z^T \Lambda Z=
    Z_{(1)}^T \Lambda_{(1)}Z_{(1)}+Z_{(2)}^T \Lambda_{(2)}Z_{(2)}
    \overset{def}{=}A+B.
    $$
      Note that $\myrank(A)=r$. By Weyl's inequality, for $i=r+1,\ldots, \min(n,p)$, we have that
    $$
    \lambda_i(B) \leq \lambda_i(Z^T \Lambda Z)\leq
    \lambda_{i-r}(B).
    $$
    Sum the above inequality over $i=r+1,\ldots, \min(n,p)$,
    $$
    \sum_{i=r+1}^{\min(n,p)}\lambda_i(B) \leq \sum_{i=r+1}^{\min(n,p)}\lambda_i(Z^T \Lambda Z)\leq
    \sum_{i=1}^{\min(n,p)-r}\lambda_{i}(B).
    $$
     It implies that
     \begin{equation*}
         \begin{aligned}
             &\Big|\sum_{i=r+1}^{n}\lambda_i(Z^T \Lambda Z)- \mytr (B)\Big|
             \leq  r \lambda_1 (B).
         \end{aligned}
     \end{equation*}
     Then
     \begin{equation}\label{smallEstimationEq1}
         \begin{aligned}
             &\Big|\sum_{i=r+1}^{n}\lambda_i(Z^T \Lambda Z)- n\sum_{i=r+1}^p \lambda_i \Big|
             \leq  r \lambda_1 (B)+\Big|\mytr (B)-n\sum_{i=r+1}^p \lambda_i\Big|.
         \end{aligned}
     \end{equation}

     Note that $\mytr(B)=\sum_{i=r+1}^p \sum_{j=1}^n \lambda_i Z_{ij}^2$, by Lemma~\ref{PCAlemma1}, with probability at least $1-2\exp(-t^2/2)$, we have
     \begin{equation}\label{smallEstimationEq2}
     \Big|\mytr(B)-n\sum_{i=r+1}^p \lambda_i\Big|\leq \sqrt{2np}Ct+ Ct^2.
     \end{equation}
     Note that $\lambda_1(B)\leq C \lambda_1(Z_{(2)}^T Z_{(2)})$. The conclusion follows by~\eqref{smallEstimationEq1},~\eqref{smallEstimationEq2} and Lemma~\ref{matrixDeviation}.


\end{proof}

