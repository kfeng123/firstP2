
\section{Model and Assumptions}


Let $\{X_{k1},\ldots,X_{kn_k}\}$, $k=1, 2$ be two independent  random samples from $p$ dimensional normal distribution with means $\mu_1$ and $\mu_2$ respectively.

\begin{assumption}\label{balance}
Assume $p\to \infty$ as $n\to \infty$. Furthermore, assume two samples are balanced, that is,
\begin{equation*}
    \frac{n_1}{n_2}\to \xi \in (0,+\infty).
\end{equation*}
\end{assumption}

To characterize correlations between $p$ variables, we consider spiked covariance structure which is adopted by PCA study. See~\cite{Cai2012Sparse} and the references given there.
\begin{assumption}\label{theModel}
Suppose $X_{ki}$, $i=1,2,\ldots,n_k$ and $k=1,2$ are generated by  following model
\begin{equation*}
X_{ki}=\mu_k+V_k D_k U_{ki}+Z_{ki},
\end{equation*}
where
$U_{ki}$'s are i.i.d.\  random vectors distributed as $r_k$ dimensional standard normal distribution with $r_k$ fixed, 
$D_k=diag(\lambda_{k1}^{\frac{1}{2}},\ldots,\lambda_{k{r_k}}^{\frac{1}{2}})$ with $\lambda_{k1}\geq \cdots \geq \lambda_{k{r_k}}>0$,
$V_k$ is  a $p\times r_k$ orthonormal matrix, $Z_{ki}$'s are i.i.d.\ random vectors distributed as  $N_p(0,\sigma^2_k I_p)$ independent of $U_{ki}$'s and $\sigma^2_k>0$, $k=1,2$.
\end{assumption}
Then $X_{ki}\sim N(\mu_k,\Sigma_k)$, where $ 
\Sigma_k=\textrm{Var}(X_{ki})=V_k\Lambda_k V_k^T+\sigma^2_k I_p
$
, $\Lambda_k=D_k^2=diag(\lambda_{k1},\ldots,\lambda_{k{r_k}})$.
From Assumption~\ref{theModel}, $V_k V_k^T$ is the orthogonal projection matrix on the column space of $V_k$. Let $\tilde{V}_k$ be a $p\times (p-r_k)$ full column rank orthonormal matrix orthogonal to columns of  $V_k$.
%, that is $\tilde{V}_k^T V_k=O_{r_k\times(p-r_k )}$
 Note that $\tilde{V}_k$ may not be unique. But the projection matrix $\tilde{V}_k\tilde{V}_k^T$ is unique because $\tilde{V}_k\tilde{V}_k^T=I-V_k V_k^T$.


\begin{assumption}\label{orderOfBeta}
    Assume that there is some constant $\kappa>0$ and $\beta\geq \frac{1}{2}$ such that
    \begin{equation*}
        \kappa p^{\beta}\geq \lambda_{k1}\geq \cdots \geq\lambda_{kr_k}\geq \kappa^{-1}p^{\beta}.
\end{equation*}
\end{assumption}


The restriction $\beta\geq 1/2$ breaks down the Condition~\eqref{chenscondition}. If $\beta< 1/2$, Condition~\eqref{chenscondition} is meet and~\cite{Chen2010A}'s  method is valid. 
 Hence $\beta=1/2$ is the boundary of the scope between $T_{CQ}$ and our new test.
The case $\beta=1$ corresponds to the factor model in paper~\cite{Ma2015A} with some restrictions of parameters.

Throughout the paper, let $\tau={(n_1+n_2)}/{(n_1n_2)}$, $S$ be the pooled sample covariance:
\begin{equation*}
S=\frac{1}{n-2}\sum_{k=1}^2\sum_{i=1}^{n_k} (X_{ki}-\bar{X}_k) {(X_{ki}-\bar{X}_k)}^T
    =\frac{(n_1-1)S_1+(n_2-1)S_2}{n-2},
\end{equation*}
where
\begin{equation*}
S_k=\frac{1}{n_k -1}\sum_{i=1}^{n_k} (X_{ki}-\bar{X}_k) {(X_{ki}-\bar{X}_k)}^T
\end{equation*}
is the sample covariance  of the sample $k$, $k=1,2$.

We write $\xi\sim \eta$ to denote the random variable $\xi$ and $\eta$ have the same distribution.
For nonrandom positive sequence $\{a_n\}$ and $\{b_n\}$, $a_n\asymp b_n$ represents $a_n\geq cb_n$ and $a_n\leq Cb_n$ for some positive $c,C$ for every $n$.

We denote by $\|\cdot \|$ and $\|\cdot\|_F$ the operator and Frobenius  norm of matrix, separately.

For a symmetric matrix $A$, we define $\lambda_i(A)$ to be the $i$th largest eigenvalue of $A$ and $\lambda_{\max}(A)$, $\lambda_{\min}(A)$ to be the maximal and minimal eigenvalues respectively.
We denote by $\mytr(A)$ the trace of $A$.

The notations $\xrightarrow{P}$ and $\xrightarrow{\mathcal{L}}$ are used to denote convergence in probability and weak convergence respectively.

Let $[m]=\{1,\ldots, m\}$.
