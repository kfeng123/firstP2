\section{PCA Theory}

We give some PCA theory here.
Compared with existing results, we impose less assumptions since our main task is to obtain the properties of principal space.

The following lemma is from~\cite{Davidson2001Local}:
\begin{lemma}[Davidson-Szarek bound]\label{matrixDeviation}
    Let $Z$ be an $N\times n$ matrix whose entries are independent standard normal random variables.
    Then for every $t> 0$, with probability at least $1-2\exp (-t^2/2)$ one has
    $$
    (\sqrt{N}-\sqrt{n}-t)^2 \leq \lambda_{\min(N,n)}(ZZ^T)\leq \lambda_{1}(ZZ^T)\leq (\sqrt{N}+\sqrt{n}+t)^2.
    $$
\end{lemma}
By the Cramer-Chernoff method, we have the following lemma.
\begin{lemma}\label{PCAlemma1}
    Under the assumption of Lemma~\ref{matrixDeviation}, then for every $t> 0$, with probability at least $1-\exp\big(-{t^2}/{2}\big)$ one has
    $$
    \mytr(ZZ^T)\geq Nn\Big(1-\sqrt{\frac{2}{Nn}}t\Big).
    $$
\end{lemma}
\begin{proof}
    Note that $\mytr(ZZ^T)\sim \chi^2_{Nn}$. Then for $t>0$, we have
\begin{equation*}
    \begin{split}
        &\Pr\big(-\mytr (ZZ^T)+ Nn \geq t\big)
        = 
        \Pr\big(\exp(-\lambda\chi^2_{Nn}+  Nn\lambda) \geq \exp (t\lambda)\big)\\
        \leq &
        \exp(\big(Nn-t\big)\lambda) \myE \exp(-\lambda \chi^2_{Nn})
        =
        \exp\Big(\big(Nn-t\big)\lambda -\frac{Nn}{2}\log(1+2\lambda)\Big),
    \end{split}
\end{equation*}
    where $\lambda>0$ can be arbitrary.
    If $0<t< Nn$, let $\lambda=\frac{t}{2(Nn-t)}$ and we get
\begin{equation*}
    \begin{split}
        &\Pr\big(-\mytr (ZZ^T)+ Nn \geq t\big)
        \leq
        \exp\Big(\frac{t}{2}+\frac{Nn}{2}\log(1-\frac{t}{Nn})\Big).
    \end{split}
\end{equation*}
    Since for $0<x<1$, $\log(1-x)\leq -x-\frac{x^2}{2}$, we have that
    \begin{equation}\label{concentrationLemmaEq}
    \begin{split}
        &\Pr\big(-\mytr (ZZ^T)+ Nn \geq t\big)
        \leq
        \exp\Big(-\frac{t^2}{4Nn}\Big).
    \end{split}
\end{equation}
    If $t\geq Nn$ The left hand side of~\eqref{concentrationLemmaEq} is $0$ for trivial reason.
    Hence~\eqref{concentrationLemmaEq} holds for all $t>0$.
    The conclusion follows by substituting $t$ by $\sqrt{2Nn}t$ in~\eqref{concentrationLemmaEq}.
\end{proof}
\begin{lemma}\label{PCAlemma2}
    Under the assumption of Lemma~\ref{matrixDeviation}, then for every $t> 0$, with probability at least $1-3\exp(-t^2/2)$, for every $i$ such that $1\leq i \leq \min(N,n)$, we have
\begin{equation*}
    \begin{split}
        \lambda_{i}(ZZ^T)
        &\geq
        \max(N,n) \Big( 1-\sqrt{\frac{2}{Nn}}t-(i-1)(\frac{1}{\sqrt{N}}+\frac{1}{\sqrt{n}}+\frac{t}{\sqrt{Nn}})^2 \Big).
    \end{split}
\end{equation*}

\end{lemma}
\begin{proof}
    By Lemma~\ref{PCAlemma1}, for every $t>0$, with probability at least $1-\exp(-t^2/2)$, we have that
    $$
    \sum_{j=1}^{i-1}\lambda_j (ZZ^T) + \sum_{j=i}^{\min(N,n)}\lambda_j (ZZ^T)=\mytr (ZZ^T)\geq Nn(1-\sqrt{\frac{2}{Nn}}t),
    $$
    where $1\leq i \leq \min(N,n)$.
    Thus, with probability at least $1-\exp(-t^2/2)$, for every $i$ such that $1\leq i \leq \min(N,n)$, we have
\begin{equation*}
    \begin{split}
        &\lambda_{i}(ZZ^T)\geq \frac{1}{\min(N,n)}\sum_{j=i}^{\min(N,n)}\lambda_j (ZZ^T)\\
        \geq &
        \frac{1}{\min(N,n)} \Big( Nn(1-\sqrt{\frac{2}{Nn}}t)-\sum_{j=1}^{i-1}\lambda_j (ZZ^T)  \Big)\\
        \geq &
        \frac{1}{\min(N,n)} \Big( Nn(1-\sqrt{\frac{2}{Nn}}t)-(i-1)\lambda_1 (ZZ^T)  \Big).
    \end{split}
\end{equation*}
    By the above inequality and Lemma~\ref{matrixDeviation}, with probability at least $1-3\exp(-t^2/2)$, for every $i$ such that $1\leq i \leq \min(N,n)$, we have
\begin{equation*}
    \begin{split}
        \lambda_{i}(ZZ^T)
        &\geq
        \frac{1}{\min(N,n)} \Big( Nn(1-\sqrt{\frac{2}{Nn}}t)-(i-1)(\sqrt{N}+\sqrt{n}+t)^2 \Big)\\
        &=
        \max(N,n) \Big( 1-\sqrt{\frac{2}{Nn}}t-(i-1)(\frac{1}{\sqrt{N}}+\frac{1}{\sqrt{n}}+\frac{t}{\sqrt{Nn}})^2 \Big).
    \end{split}
\end{equation*}
    The last equality holds since $\max(N,n)=Nn/\min(N,n)$.

\end{proof}

\begin{assumption}\label{PCAassump}
    Suppose that $Z=(Z_1,\ldots,Z_n)$ is an $p\times n$ random matrix whose entries $Z_{ij}$'s are i.i.d. standard normal random variables, $ i=1,\ldots, p$, $j=1,\ldots, n$.
    Let the sample matrix be $X=(X_1,\ldots,X_n)=U\Lambda^{1/2}Z$, where $\Lambda=\mydiag(\lambda_1,\ldots,\lambda_p)$ with $\lambda_1\geq\cdots\geq \lambda_p$ and $U$ is a $p\times p$ orthogonal matrix.
    Suppose  $c\leq \lambda_{p} \leq \lambda_{r+1}\leq C$, where $c>0$ and $C>0$ are absolute constants. 
\end{assumption}
    The sample covariance matrix is $\frac{1}{n}X X^T$.
    We denote by  $\frac{1}{n}X X^T=\hat{U}\hat{\Lambda}\hat{U}^T$ the spectral decomposition of $S$ where $\hat{\Lambda}=\mathrm{diag}(\hat{\lambda}_1,\ldots,\hat{\lambda}_p)$ and $\hat{U}$ is a orthogonal matrix.
    
    Let $u_i$ be the $i$th column of $U$, $i=1,\ldots, p$. Denote $U=(V,\tilde{V})$, where $V$ and $\tilde{V}$ are the first $r$ and last $p-r$ columns of $U$ respectively. Similarly, we define the corresponding part of $\hat{U}$ by $\hat{u}_i$, $\hat{V}$ and $\hat{\tilde{V}}$.

    The PCA theory is mainly focus on the convergence properties of $\hat{u}_i$ to it's population counterpart $u_i$.
    See~\cite{Jung2009PCA},~\cite{2012arXiv1211.2671S},~\cite{Shen2013Consistency} and~\cite{Fan2015Asymptotics} for some recent developements for PCA theory.
    Here we are mainly interested in the asymptotic properties of $\hat{V}$.
    Compared to existing results, the consistency results of $\hat{V}$ require less assumptions on the order of $\lambda_1,\ldots,\lambda_r$.

    Let $Z_{(1)}$ and $Z_{(2)}$ be the first $r$ rows and the last $p-r$ rows of $Z$ respectively. Then $Z_{(1)}$ is an $r\times n$ matrix and $Z_{(2)}$ is an $(p-r)\times n$ matrix.
    Let $\Lambda_{(1)}=\mathrm{diag}(\lambda_1,\ldots,\lambda_r)$ and $\Lambda_{(2)}=\mathrm{diag}(\lambda_{r+1},\ldots,\lambda_p)$.
    Define $\hat{\Lambda}_{(1)}$ and $\hat{\Lambda}_{(2)}$ in a similar way.


\begin{theorem}
    Suppose Assumption~\ref{PCAassump} holds.
    Let $i$ be a fixed number such that $1 \leq i\leq r$, then for every $t>0$, with probability at least $1-9\exp(-t^2/2)$, we have
    \begin{equation*}
        \begin{split}
            &\frac{\hat{\lambda}_i}{\lambda_i}\geq 
        1-\frac{2}{\sqrt{n}}(\sqrt{r}+t)+\\
            \frac{c\max(p-r,n)}{n\lambda_i} &\Big( 1-\sqrt{\frac{2}{(p-r)n}}t-(i-1)\big(\frac{1}{\sqrt{p-r}}+\frac{1}{\sqrt{n}}+\frac{t}{\sqrt{(p-r)n}}\big)^2 \Big),\\
        \end{split}
    \end{equation*}
    and
    \begin{equation*}
        \begin{split}
            \frac{\hat{\lambda}_i}{\lambda_i}\leq 
        1+\frac{2}{\sqrt{n}}(\sqrt{r}+t)+\frac{1}{n}(\sqrt{r}+t)^2+
        \frac{C}{n\lambda_i} (\sqrt{p-r}+\sqrt{n}+t)^2.
        \end{split}
    \end{equation*}

    %\begin{equation*}
       %L_1+L_2 \leq\frac{\hat{\lambda}_i}{\lambda_i}\leq U_1+U_2
    %\end{equation*}
    %where 
    %\begin{align*}
        %L_1&=1-\frac{2}{\sqrt{n}}(\sqrt{r}+t),\\
        %L_2&=\frac{c\max(p-r,n)}{n\lambda_i} \Big( 1-\sqrt{\frac{2}{(p-r)n}}t-(i-1)\big(\frac{1}{\sqrt{p-r}}+\frac{1}{\sqrt{n}}+\frac{t}{\sqrt{(p-r)n}}\big)^2 \Big),\\
        %U_1&=1+\frac{2}{\sqrt{n}}(\sqrt{r}+t)+\frac{1}{n}(\sqrt{r}+t)^2,\\
        %U_2&=\frac{C}{n\lambda_i} (\sqrt{p-r}+\sqrt{n}+t)^2.
    %\end{align*}

\end{theorem}
\begin{proof}
    The non-zero eigenvalues of $\frac{1}{n}XX^T$ are equal to that of $\frac{1}{n}X^T X$.
    And $\frac{1}{n}X^T X$ can be further written as the sum of two quantities
    \begin{equation*}
        \frac{1}{n}X^T X=\frac{1}{n}Z^T \Lambda Z=\frac{1}{n}Z_{(1)}^T \Lambda_{(1)} Z_{(1)} +\frac{1}{n}Z_{(2)}^T \Lambda_{(2)} Z_{(2)} \overset{def}{=}A+B.
    \end{equation*}
    By Weyl's inequality,
    \begin{equation}\label{PCA19}
        \frac{\max(\lambda_i(A),\lambda_i(B))}{\lambda_i}\leq\frac{\hat{\lambda}_i}{\lambda_i}\leq \frac{\lambda_i(A)}{\lambda_i} +\frac{\lambda_{\max}(B)}{\lambda_i},
    \end{equation}
    where $i=1,\ldots, r$. We deal with $\lambda_i(A)$ and $\lambda_i(B)$ separately.

    First we deal with ${\lambda_i(A)}$, $ i=1,\ldots, r$.
    By Corollary~\ref{WeylCor}, we have that
    \begin{equation*}
        \begin{aligned}
            \frac{\lambda_i(A)}{\lambda_i}&\leq
            \frac{1}{n\lambda_i}\lambda_{\max}\big(Z_{(1)}^T \mydiag(\underbrace{0,\ldots,0}_{i-1},\underbrace{\lambda_i,\ldots,\lambda_i}_{r-i+1}) Z_{(1)} \big).
        \end{aligned}
    \end{equation*}
    Then by Lemma~\ref{matrixDeviation}, with probability at least $1-2\exp(-t^2/2)$ we have 
    \begin{equation}\label{eigenvalueTheorem:2}
    \frac{\lambda_i(A)}{\lambda_i}\leq
    \frac{1}{n}{(\sqrt{n}+\sqrt{r-i+1}+t)}^2
    \leq
    1+\frac{2}{\sqrt{n}}(\sqrt{r}+t)+\frac{1}{n}(\sqrt{r}+t)^2.
\end{equation}
    On the other hand, by Weyl's inequility, we have that
\begin{equation*}
    \begin{aligned}
        \frac{\lambda_i(A)}{\lambda_i}&\geq
        \frac{1}{n\lambda_i}\lambda_{i}\big(Z_{(1)}^T \mathrm{diag}(\underbrace{\lambda_i,\ldots,\lambda_i}_{i},\underbrace{0,\ldots,0}_{r-i}) Z_{(1)} \big).\\
    \end{aligned}
\end{equation*}
    Again by Lemma~\ref{matrixDeviation}, with probability at least $1-2\exp(-t^2/2)$ we have 
\begin{equation}\label{eigenvalueTheorem:3}
        \frac{\lambda_i(A)}{\lambda_i}\geq
        \frac{1}{n}{(\sqrt{n}-\sqrt{i}-t)}^2
        \geq 1-\frac{2}{\sqrt{n}}(\sqrt{r}+t).
\end{equation}

    Now we deal with $\lambda_{i}(B)$. Since $\lambda_i(B)\geq \frac{c}{n} \lambda_i(Z_{(2)}^T Z_{(2)})$, by Lemma~\ref{PCAlemma2}, with probability at least $1-3\exp(-t^2/2)$ we have
\begin{equation}\label{eigenvalueTheorem:4}
    \begin{split}
        \lambda_{i}(B)
        &\geq
        \frac{c\max(p-r,n)}{n} \Big( 1-\sqrt{\frac{2}{(p-r)n}}t-(i-1)(\frac{1}{\sqrt{p-r}}+\frac{1}{\sqrt{n}}+\frac{t}{\sqrt{(p-r)n}})^2 \Big).
    \end{split}
\end{equation}
    Since $\lambda_1(B)\leq \frac{C}{n}\lambda_1(Z_{(2)}^T Z_{(2)})$, by Lemma~\ref{matrixDeviation}, with probability at least $1-2\exp(-t^2/2)$ we have
\begin{equation}\label{eigenvalueTheorem:5}
    \begin{split}
        \lambda_{\max}(B)
        &\leq
        \frac{C}{n} (\sqrt{p-r}+\sqrt{n}+t)^2.
    \end{split}
\end{equation}
    The theorem follows by~\eqref{PCA19},~\eqref{eigenvalueTheorem:2},~\eqref{eigenvalueTheorem:3},~\eqref{eigenvalueTheorem:4}, and \eqref{eigenvalueTheorem:5}.
\end{proof}



\begin{theorem}
    Suppose Assumption~\ref{PCAassump} holds and $p/n\to \infty$.
    If  $\frac{p}{n\lambda_r}\to 0$, then almost surely we have

    %\begin{equation}\label{PCAINeedThis}
    %    \mathrm{tr}\tilde{V}^T \hat{V}\Lambda_{(1)}\hat{V}^T \tilde{V}\asymp \frac{p}{n}
    %\end{equation}
%and
    \begin{equation}\label{PCAtheorem101}
        \|\hat{V}\hat{V}^T-VV^T\|^2_F\asymp\frac{p}{n\lambda_r }.
    \end{equation}
    If  $\frac{p}{n\lambda_r}\to \infty$, then
    \begin{equation}\label{PCAtheorem102}
        r-\frac{1}{2}\|\hat{V}\hat{V}^T-VV^T\|^2_F=O_{a.s.}(\frac{n\lambda_1}{p}).
    \end{equation}
\end{theorem}


\begin{proof}

    Since
    \begin{equation*}
        \frac{1}{n}XX^T=\hat{U}\hat{\Lambda}\hat{U}^T=
        \frac{1}{n}U\Lambda^{1/2}ZZ^T \Lambda^{1/2} U^T,
    \end{equation*}
    we have
    \begin{equation}\label{crucialEqInPCA1}
        \Lambda^{-1/2}U^T \hat{U}\hat{\Lambda}\hat{U}^T U\Lambda^{-1/2}=
        \frac{1}{n}ZZ^T. 
    \end{equation}
    First, we prove~\eqref{PCAtheorem101}.
    It follows from~\eqref{crucialEqInPCA1} that
    \begin{equation}\label{crucialEqInPCA3}
        \Lambda^{-1/2}_{(2)}\tilde{V}^T \hat{U}\hat{\Lambda}\hat{U}^T \tilde{V}\Lambda^{-1/2}_{(2)}=
        \frac{1}{n}\tilde{Z}_{(2)}\tilde{Z}_{(2)}^T.
    \end{equation}

    The left hand side of~\eqref{crucialEqInPCA3} equals to $C+D$, where 
    $C= \Lambda^{-1/2}_{(2)}\tilde{V}^T \hat{V}\hat{\Lambda}_{(1)}\hat{V}^T \tilde{V}\Lambda^{-1/2}_{(2)}$ 
    and 
    $D= \Lambda^{-1/2}_{(2)}\tilde{V}^T \hat{\tilde{V}}\hat{\Lambda}_{(2)}\hat{\tilde{V}}^T \tilde{V}\Lambda^{-1/2}_{(2)}$.

    Obviously, we have $\lambda_1(C)\leq {n^{-1}} \lambda_1(\tilde{Z}_{(2)} \tilde{Z}_{(2)}^T)$ and
    %$$
    %\frac{\hat{\lambda}_r}{C} \lambda_1 (\tilde{V}^T \hat{V}\hat{V}^T \tilde{V}) \leq \lambda_1(C)\leq \frac{\hat{\lambda}_1}{c} \lambda_1 (\tilde{V}^T \hat{V}\hat{V}^T \tilde{V}).
    %$$
    $
    \lambda_1(C)   \geq {C^{-1}}{\hat{\lambda}_r} \lambda_1 (\tilde{V}^T \hat{V}\hat{V}^T \tilde{V})
    $.
    It follows that 
    $$
    \lambda_1 (\tilde{V}^T \hat{V}\hat{V}^T \tilde{V})\leq 
    \frac{C}{n\hat{\lambda}_r} \lambda_1(\tilde{Z}_{(2)}\tilde{Z}_{(2)}^T).
    $$
    

    If $p-r\geq n$, then $\mathrm{Rank}(C)=r$, $\mathrm{Rank}(D)=n-r$ and $\mathrm{Rank}(C+D)=n$.
    By Weyl's inequality,
    $$\lambda_n(C+D)\leq \lambda_r(C)+ \lambda_{n-r+1}(D)=\lambda_r(C).$$
    Thus, 
    $$
 \frac{1}{n}\lambda_n(\tilde{Z}_{(2)}\tilde{Z}_{(2)}^T)   \leq \lambda_r(C)
    \leq  \frac{\hat{\lambda}_1}{c} \lambda_r(\tilde{V}^T \hat{V}\hat{V}^T \tilde{V}),
    $$
    or
    $$
      \lambda_r(\tilde{V}^T \hat{V}\hat{V}^T \tilde{V})
    \geq
    \frac{c}{n\hat{\lambda}_1}\lambda_n(\tilde{Z}_{(2)}\tilde{Z}_{(2)}^T).
    $$

    By Bai Yin's law, we have that
    \begin{equation*}
        \lambda_1\big(\frac{1}{p}\tilde{Z}_{(2)}\tilde{Z}_{(2)}^T\big)\to 1,\quad
        \lambda_{n-1}\big(\frac{1}{p}\tilde{Z}_{(2)}\tilde{Z}_{(2)}^T\big)\to 1\quad  a.s..
    \end{equation*}
    By Lemma~\ref{lemmaRankLim}, $\lambda_{1}(C)\xrightarrow{a.s.}1$ and $\lambda_{r}(C)\xrightarrow{a.s.}1$. It follows that
    \begin{equation}\label{PCA27eq}
        \frac{n}{p}\hat{\Lambda}_{(1)}^{1/2}\hat{V}^T \tilde{V}\Lambda^{-1}_{(2)}\tilde{V}^T \hat{V}\hat{\Lambda}_{(1)}^{1/2}\xrightarrow{a.s.} I_r.
    \end{equation}
    When $\frac{p}{n\lambda_r}\to 0$, $\hat{\lambda}_i$'s are ratio consistent for $1\leq i\leq r$. That is, $\Lambda_{(1)}^{-1}\hat{\Lambda}_{(1)}\to I_r$ almost surely. Then it follows from~\eqref{PCA27eq} that
    \begin{equation}
        \frac{n}{p}\Lambda_{(1)}^{1/2}\hat{V}^T \tilde{V}\Lambda^{-1}_{(2)}\tilde{V}^T \hat{V}\Lambda_{(1)}^{1/2}\xrightarrow{a.s.} I_r.
    \end{equation}

    %Since $\Lambda_{(2)}$ is bounded from below and above,~\eqref{PCAINeedThis} holds.
    Notice that
    \begin{equation*}
        \begin{aligned}
         \frac{n}{p}   \mathrm{tr}\big(\Lambda_{(1)}^{1/2}\hat{V}^T \tilde{V}\Lambda^{-1}_{(2)}\tilde{V}^T \hat{V}\Lambda_{(1)}^{1/2}\big)&\geq
          \frac{n}{p}  \lambda_r\mathrm{tr}\big(\hat{V}^T \tilde{V}\Lambda^{-1}_{(2)}\tilde{V}^T \hat{V}\big)
            \geq
          \frac{n}{p}  e_r^T \hat{\Lambda}_{(1)}^{1/2}\hat{V}^T\tilde{V}\Lambda_{(2)}^{-1}\tilde{V}^T\tilde{V}\hat{\Lambda}_{(1)}^{1/2}e_1
        \end{aligned}
    \end{equation*}
    where $e_r=(\underbrace{0,\ldots,0}_{r-1},1)$. It follows that the medium term is bounded above and below asymptotically. Notice that
    \begin{equation*}
        \begin{aligned}
            \frac{n}{p}\lambda_r\mathrm{tr}\big(\hat{V}^T \tilde{V}\Lambda^{-1}_{(2)}\tilde{V}^T \hat{V}\big)
            &\asymp
            \frac{n}{p}\lambda_r\mathrm{tr}\big(\hat{V}^T \tilde{V}\tilde{V}^T \hat{V}\big)
            =\frac{n}{p}\lambda_r\frac{1}{2}\|VV^T -\hat{V}\hat{V}^T\|^2_F.
        \end{aligned}
    \end{equation*}
     Therefore
    $\|VV^T -\hat{V}\hat{V}^T\|^2_F\asymp \frac{p}{n\lambda_r}$ almost surely.

    Then we prove~\eqref{PCAtheorem102}. It follows from~\eqref{crucialEqInPCA1} that
    \begin{equation}\label{crucialEqInPCA2}
        \Lambda^{-1/2}_{(1)}V^T \hat{U}\hat{\Lambda}\hat{U}^T V\Lambda^{-1/2}_{(1)}=
        \frac{1}{n}\tilde{Z}_{(1)}(I-\frac{1}{n}J)\tilde{Z}_{(1)}^T\xrightarrow{a.s.} I_{r}.
    \end{equation}
But
    \begin{equation}
        \begin{aligned}
        \mathrm{tr}\big(\Lambda^{-1/2}_{(1)}V^T \hat{U}\hat{\Lambda}\hat{U}^T V\Lambda^{-1/2}_{(1)}\big)
            &\geq
        \mathrm{tr}\big(\Lambda^{-1/2}_{(1)}V^T \hat{V}\hat{\Lambda}_{(1)}\hat{V}^T V\Lambda^{-1/2}_{(1)}\big)
            \\
            &\geq
            \frac{\hat{\lambda}_r}{\lambda_1}\Big(r-\frac{1}{2}\|\hat{V}\hat{V}^T-VV^T\|^2_F\Big).
        \end{aligned}
    \end{equation}
When $\frac{p}{n\lambda_r}\to \infty$, $\hat{\lambda}_r\asymp p/n$.Then~\eqref{PCAtheorem102} holds.

\end{proof}



Suprisingly, from our proof we can see that the error of PCA can be estimated well!
