
\section{Numerical studies}
\subsection{Simulation results}

Our simulation study focus on equal variance case. 
We generate $X_{ki}$ by the model in Assumption~\ref{theModel}, where each element of $U_{ki}$ and $Z_{ki}$ are generated from $N(0,1)$.
$V$ is a random orthonormal matrix. 
We generate $\lambda_i$ as $p^{\beta}$ plus a random error from $U(0,1)$.

%The key to the validation of Theorem~\ref{myPanpan} is  that $T_{\textrm{dif}}=\frac{n_1n_2|T_1-T_2|}{\sqrt{2p}(n_1+n_2)\sigma^2}$ converges to $0$.
%Here we verify it by simulation.
%We set $n_1=n_2=n$, $p=n^i$ for $i=1,2$ and plot $T_{\textrm{dif}}$ versus $p$.
%The results are illustrated in figure~\ref{fig:fig1}.
%From the results we can find that $T_{\textrm{dif}}$ clearly converges to $0$ when $p=n$.
%In the case of $p=n^2$ which is exactly beyond the assumption of Theorem~\ref{myPanpan},
%$T_{\textrm{dif}}$ is large and it's not clear whether $T_{\textrm{dif}}$  converges to $0$.
%\begin{figure}
%    \centering 
%    \includegraphics[height=6cm]{code/difference1.jpeg}
%    \includegraphics[height=6cm]{code/difference2.jpeg}\\
%    \includegraphics[height=6cm]{code/difference3.jpeg}
%    \includegraphics[height=6cm]{code/difference4.jpeg}\\
%    \includegraphics[height=6cm]{code/difference5.jpeg}
%    \includegraphics[height=6cm]{code/difference6.jpeg}\\
%    \caption{These are plots of $T_{\textrm{dif}}$ versus $p$. The first column and the second column are the case of $p=n$ and $p=n^2$, separately. The cases of $\beta=1,2,3$ are in the row $1,2,3$ separately. $r$ is set to be $3$ in all cases. }\label{fig:fig1}
%\end{figure}

First we simulate the level of the new test. The nominal level $\alpha=0.05$ and we set $r=2$. Samples are repeatedly generated $1000$ times to calculate empirical level.  For comparison, we also give corresponding `oracle' level which is calculated by `statistic' ${T_1}/(\sigma^2\sqrt{2p\tau^2})$ whose asymptotic normality can be guaranteed by Theorem 1 in~\cite{Chen2010A}. The results are listed in
Table~\ref{biaoge1}. From the results, we can find that for small $n$ and $p$, even oracle level is not satisfied. Level of the new test is  a little inflated compared with oracle level and it performs better when $n$ is larger.

\input{code/level.tex}



Then we simulate the empirical power of our test and~\cite{Chen2010A}'s test. The simulation results of~\cite{Ma2015A} have showed that the level of the~\cite{Chen2010A}'s test can't be guaranteed when covariance is spiked. To be fair, we use permutation method to compute critical value. The validity of permutation method can be found in~\cite{Lehmann}'s Example 15.2.2. We plot the empirical power versus $\|\mu_1-\mu_2\|$ when other parameters hold constant. The results are illustrated in figure~\ref{fig:fig2}.
From the results, we can find that when $\Sigma$ is spiked, the new test outperforms $T_{CQ}$ substantially; when $\Sigma$ is not spiked, the new test and $T_{CQ}$ are comparable.
\begin{figure}
    \centering 
    \includegraphics[height=6cm]{code/fig1.jpeg}
    \includegraphics[height=6cm]{code/fig2.jpeg}
    \\
    \includegraphics[height=6cm]{code/fig3.jpeg}
    \includegraphics[height=6cm]{code/fig4.jpeg}
    \\
    \includegraphics[height=6cm]{code/fig5.jpeg}
    \includegraphics[height=6cm]{code/fig6.jpeg}
    \caption{Empirical power simulation. $\alpha$ is set to be $0.05$. $d$ is proportional to $\|\mu_1-\mu_2\|^2$. For each simulation, we do 50 permutations to determine critical value. We generate $100$ independent samples to compute empirical power. }\label{fig:fig2}
\end{figure}

%Permutation method is computation expensive. So when $p$ and $n$ are large, we simulate empirical power by asymptotic distribution. The results are illustrated in figure~\eqref{fig:fig3}.

%\begin{figure}\label{fig:fig3}
    %\centering 
    %\includegraphics[height=6cm]{code/newfig1.jpeg}
    %\includegraphics[height=6cm]{code/newfig2.jpeg}
    %\\
    %\includegraphics[height=6cm]{code/newfig3.jpeg}
    %\includegraphics[height=6cm]{code/newfig4.jpeg}
    %\\
    %\includegraphics[height=6cm]{code/newfig5.jpeg}
    %\includegraphics[height=6cm]{code/newfig6.jpeg}
    %\caption{Empirical Power (critical values are computed by asymptotic distribution)}\label{fig:fig3}
%\end{figure}

\subsection{Real data analysis}
In this section, we study the same practical problem as~\cite{Ma2015A} did. That is testing whether Monday stock returns are equal to those of other trading days on average. Define an observation be the log return of stocks in a day. Hence $p$ is the total number of stocks. Let sample $1$ and sample $2$ be the observations on Monday and the other trading days, respectively.  Then we would like to test $H_0\, :\mu_1=\mu_2$ v.s. $H_1\,:\mu_1\neq \mu_2$. We collected the data of $p=710$
 stocks of China
from 01/04/2013 to 12/31/2014. There are total $n_1=95$ Monday and $n_2=388$ other trading days. 

We assume $\Sigma_1=\Sigma_2$. The first eigenvaule of $S$ is $0.14$, which is significantly larger than the others.
In fact, the second eigenvalue is $0.02$.
Hence there's clearly a spiked eigenvalue. We set $r=1$ and perform our new test. The $p$ value is $0.149$, which is obtained by $1000$ permutations. Hence, the null hypothesis can not be rejected for $\alpha=0.05$. We draw the same conclusion as~\cite{Ma2015A}.

